% Thanks to Seth Johnson (sethrj@umich.edu) for publishing his ans template

%%%%%%%%%%%%%%%%%%%%%%%%%%%%%%%%%%%
\documentclass{anstrans}
\bibliographystyle{ans}

%%%%%%%%%%%%%%%%%%%%%%%%%%%%%%%%%%%
\title{An Agent-Based Framework for Fuel Cycle Simulation with Recycling}
\author{Matthew J.~Gidden, Paul P.H.~Wilson, Kathryn D.~Huff, Robert W.~Carlsen}
\institute{Department of Nuclear Engineering \& Engineering Physics, University of Wisconsin - Madison, Madison, WI, 53703}
\email{gidden@wisc.edu}
\date{2013/01/14}

%%%%%%%%%%%%%%%%%%%%%%%%%%%%%%%%%%%
\usepackage{times} % fancier looking type
\usepackage{graphicx}
\usepackage{microtype} % if using PDF
\usepackage{amsmath} % for optimization equations
\usepackage{booktabs}

%%%%%%%%%%%%%%%%%%%%%%%%%%%%%%%%%%%%%%%%%%%%%%%%%%%%%%%%%%%%%%%%%%%%%%%
\begin{document}

%%%%%%%%%%%%%%%%%%%%%%%%%%%%%%%%%%%%%%%%%%%%%%%%%%%%%%%%%%%%%%%%%%%%%%%
\section{Introduction}
There is strong motivation for pursuing a decision-based, 
discrete-time, discrete-object fuel cycle simulation infrastructure. 
The fuel cycle simulation community has long been a collection of 
individual actors designing simulation tools for specific 
requirements, the results of which are then difficult or impossible 
to compare/benchmark. Additional difficulties arise when proprietary 
software is used as the foundation of the model framework due to 
licensing restrictions and high costs. Accordingly, we have been 
working through a number of design iterations to arrive at a generic 
framework that supports the nuclear systems simulation community in a 
broad sense, i.e. a ``generic-enough'' tool to allow for system design, 
connection, and analysis as well as comparison with other similarly 
designed systems.

When viewing the nuclear fuel cycle through a facility-connections 
lens, it becomes apparent that the life cycle of material constitutes 
a supply chain. Different supply chain structures are currently 
present in the real fuel cycle, e.g. the rigid top-down structure in France 
as opposed to the more global, multi-tiered supply chains resulting 
from U.S. Section 123 Agreements. In order to model the variety of
possible structures, one must allow for independent decision making. 
Due to the aggregation of requirements, a natural fit is the 
burgeoning field of agent-based, supply-chain network simulation.

%%%%%%%%%%%%%%%%%%%%%%%%%%%%%%%%%%%%%%%%%%%%%%%%%%%%%%%%%%%%%%%%%%%%%%%
\section{Simulation Framework}
There are a number of agent-based supply chain frameworks and 
implementations available in the literature with varying levels of 
accessibility due to proprietary considerations.\cite{swaminathan_modeling_1998}\cite{julka_agent-based_2002}\cite{van_der_zee_modeling_2005}\cite{chatfield_multi-formalism_2007}
However, the nuclear fuel cycle presents a few unique characteristics not explicitly 
treated in the literature. Perhaps the most difficult consideration 
we have identified is the need to specify target fuel recipes and
match suppliers and consumers based on the requested recipe, i.e. 
there are both quantity and quality constraints placed on a requested 
commodity. An additional difficulty arises with the enforcement of 
regional-boundary constraints (e.g. prohibiting HEU trade between 
regions) and inter-enterprise preferences. We propose to tackle both 
issues via a comprehensive supply/demand matching mechanism.

We propose using a supply/demand matching algorithm that is comprised 
of three main procedures: request-for-bids, preference assignment, 
and resolution. The request-for-bids step signals the producers of 
various commodities of the demand and material specification for 
those commodities. The preference assignment step allows the customers
to analyze each bid in order to assign a preference. The managers of 
these customers (be they at the region or enterprise level) are 
allowed to affect the decision making process at this point in order 
to inform the preferences of the customers covered by their policy 
space. The resolution step takes the as input the bids and 
preferences and outputs the material flows for the given time step.\cite{cyclus2012}

%%%%%%%%%%%%%%%%%%%%%%%%%%%%%%%%%%%%%%%%%%%%%%%%%%%%%%%%%%%%%%%%%%%%%%%
\section{Conclusions}
Presented herein are a number of simple simulations in order to 
demonstrate the capability of an implementation of the framework. We 
discuss various modeling tradeoffs encountered while constructing 
once-through and recycle simulation cases. We finish with a 
discussion of the future simulation landscape and expansion of the 
use of the framework.

%%%%%%%%%%%%%%%%%%%%%%%%%%%%%%%%%%%%%%%%%%%%%%%%%%%%%%%%%%%%%%%%%%%%%%%
\section{Acknowledgments}
This research is being performed using funding received from the DOE
Office of Nuclear Energy's Nuclear Energy University Programs.  The
author thanks the NEUP for its generous support.

%%%%%%%%%%%%%%%%%%%%%%%%%%%%%%%%%%%%%%%%%%%%%%%%%%%%%%%%%%%%%%%%%%%%%%%
\bibliography{bibliography}
\end{document}
