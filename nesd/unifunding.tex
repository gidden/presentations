%||||---------------
\begin{frame}[ctb!]
  \frametitle{University Programs: NEUP}
  The Nuclear Energy University Partnership ``consolidates [the DOE-NE's] 
  university support under one program.''
  \vspace{0.4cm}
  \pause
  
  The NEUP funding pool
  \begin{itemize}
    \item is \emph{up to 20\%} of the DOE-NE's R\&D budget (\textasciitilde\$60M)
    \item for infrastructure upgrade and maintenance
    \item for mission-critical R\&D
  \end{itemize}
  \vspace{0.3cm}
  \pause
  
  The NEUP \emph{administrates} but \emph{does not provide funding} for
  DOE-NE funded scholarships and fellowships.
\end{frame}
%---------------||||

%||||---------------
\begin{frame}[ctb!]
  \frametitle{University Programs: IUP}
  The Integrated University Program provides funding for:
  \begin{itemize}
    \item 4-year, 2-year, and technical schools
      \begin{itemize}
        \item Scholarships
        \item Fellowships
      \end{itemize}
    \item New Faculty Grants
    \item Curriculum Development
    \item Non-mission Related Research
  \end{itemize}
  \pause
  Spread across three government bodies:
  \begin{itemize}
    \item DOE Office of Nuclear Energy (\$5M FY2012)
      \begin{itemize}
        \item Scholarships \& Fellowships
        \item Faculty \& Curriculum Development
      \end{itemize}
    \item DOE NNSA (\$15M FY2012)
      \begin{itemize}
        \item UC-Berkely Security Consortium
        \item General R\&D
      \end{itemize}
    \item NRC (\$4.7M FY2012)
      \begin{itemize}
        \item Faculty \& Curriculum Development
       \end{itemize}
  \end{itemize}
\end{frame}
%---------------||||

%||||---------------
\begin{frame}[ctb!]
  \frametitle{University Programs: IUP History}
  Outlined in the Energy and Water Appropriations Act of
  2009, the IUP was to fund a program for \$450M over 10 
  years evenly split between (authorized to be appropriated):
  \begin{itemize}
    \item DOE - NE
    \item NRC
    \item DOE - NNSA
  \end{itemize}
  To be split: \$10M - mission related; \$5M discipline related
  \vspace{0.4cm}
  \pause
  
  During the continuing resolution, the FY2012 Enacted budget,
  specified by Congress, \emph{appropriates}:
  \begin{itemize}
    \item DOE - NE: \$5M
    \item NRC: \$4.7M
    \item DOE - NNSA: \$15M
  \end{itemize}
\end{frame}
%---------------||||


%||||---------------
\begin{frame}[ctb!]
  \frametitle{University Programs: DOE Language}  
  From the DOE's FY2013 budget proposal \cite{doe_budget_2013}:
  \begin{quote}
    In FY 2011, DOE provided no funding in its Operating Plan 
    for the Integrated University Program (IUP) and no funding 
    is being requested in FY 2013 for the program.  IUP has 
    consistently been proposed for termination.  This program is 
    a less efficient means to advance the Administration’s STEM 
    objectives than other existing programs.  In addition, as the 
    nuclear industry expands, it will create the incentives 
    necessary for students to enter nuclear-related programs. 
    Although no funding was requested in FY 2012, \$5 million was 
    congressionally-directed for IUP.  Funding was used to support 
    nuclear science and engineering study and research through 
    scholarships and fellowships. 
  \end{quote}
\end{frame}
%---------------||||

%||||---------------
\begin{frame}[ctb!]
  \frametitle{University Programs: NRC Language}  
  From the NRC's FY2013 budget proposal \cite{nrc_budget_2013}:
  \begin{quote}
    Resources also support \$5 million for grants to universities 
    for university-led, mission-related support (curriculum development) 
    for nuclear science, engineering, and related disciplines and trades.

    However, no funding is requested for the Integrated University Program.  
    This reflects the confidence that the nuclear industry, as it expands, 
    will create incentives for students to enter nuclear-related programs.
  \end{quote}
\end{frame}
%---------------||||

%||||---------------
\begin{frame}[ctb!]
  \frametitle{University Programs: Industry Language}  
  From NEI \cite{nei_statement}:
  \begin{quote}
    In 2011, the industry contributed more than \$15 million to 
    universities and community colleges to support nuclear engineering 
    programs (\$8.6 million) and nuclear energy technician programs 
    (\$6.6 million). The industry’s direct support included cash grants, 
    scholarships, fellowships, equipment donations, internships, co-ops 
    and subject-matter expert support.
  \end{quote}
\end{frame}
%---------------||||
