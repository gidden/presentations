% Thanks to Seth Johnson (sethrj@umich.edu) for publishing his ans template

\documentclass{anstrans}
\bibliographystyle{ans}

%%%%%%%%%%%%%%%%%%%%%%%%%%%%%%%%%%%
\title{Once-Through Bechmarks with CYCLUS, a Modular, Open-Source Fuel Cycle Simulator}
\author{Matthew J.~Gidden, Paul P.H.~Wilson, Kathryn D.~Huff, Robert W.~Carlsen}

%% uncomment these next five only if using anstrans
\institute{Department of Nuclear Engineering \& Engineering Physics, University of Wisconsin - Madison, Madison, WI, 53703}
\email{gidden@wisc.edu}
\usepackage{graphicx}
\usepackage{microtype} % if using PDF
\newcommand{\units}[1] {\:\text{#1}}%
\newcommand{\SN}{S$_N$}%{S$_\text{N}$}%{$S_N$}%

\date{2012/06/29}
%%%%%%%%%%%%%%%%%%%%%%%%%%%%%%%%%%%
\begin{document}

%%%%%%%%%%%%%%%%%%%%%%%%%%%%%%%%%%%%%%%%%%%%%%%%%%%%%%%%%%%%%%%%%%%%%%%%%%%%%%%%
\section{Introduction}
The Cyclus project at the University of Wisconsin - Madison is the result of previous lessons learned and is designed to be a 
modular application, allowing for a variety of collaborators and a potentially diverse user-base. Cyclus is written in C++ and
is composed of a core library which is linked to a selection of module and utility libraries. In its current form, Cyclus uses
XML-based pre-processing and a SQL-based output database. The Cyclus team has recently grown and now incorporates a variety of 
expertise: output visualization capability through collaboration with the University of Utah, server-client communication via the 
University of Idaho, input visualization and control with the University of Texas - Austin, and social communication expertise 
through collaborators at UW-Madison to assist the mission-critical goal of relevancy vis-\`{a}-vis policy makers. Accordingly, the 
Cyclus project is expanding efforts in the realms of both structural capability and benchmarking calculations.

A series of once-through fuel cycle scenarios have been conducted using the Cyclus core and accompanying modules. Where needed,
additional modules have been added to the basick pack of Cyclus module, including a reactor module that operates in a batch 
mode and an region model that intelligently makes building decisions given a demand function. The results of these scenarios
are then compared with current industry standard simulators, such as VISION \cite{vision2009}, to provide a benchmark of the
Cyclus results.
%%%%%%%%%%%%%%%%%%%%%%%%%%%%%%%%%%%%%%%%%%%%%%%%%%%%%%%%%%%%%%%%%%%%%%%%%%%%%%%%
\section{Cyclus Design and Development}
The paradigm under which Cyclus has been developed and is distributed is unique and offers a number of advantages regarding its 
potential wide-spread adoption by a diverse user base. Cyclus is an open source collection of libraries that
allow for both out-of-the-box capability and a platform on which to develop additional modules. At present, the Cyclus suite 
includes a core library, a basic module pack and utilities library, and an optimization library. Each can be linked to an 
application to be used in concert.
%%%%%%%%%%%%%%%%%%%%%%%%%%%%%%%%%%
\subsection{Open Development}
As one of the leading principles guiding the development of Cyclus, an open-source repository provides a high degree of flexibility 
and a large amount of exposure to potential collaborators and developers. The Cyclus repository is publicly available via 
GitHub \cite{cyclus2012}. A number of tools are available to Cyclus developers in order to maintain software development best-practices, 
including distributed version control, automatic documentation, and in-depth issue management. Additionally, Cyclus developers have taken 
advantage of existing, external open-source libraries in order to conform to current standards, including an expansion of the C++ standard
library and fully benchmarked linear and integer programming solvers. Perhaps the most important attribute of the open development paradigm 
is that it allows for unfettered access to the Cyclus core and basic module source code. Combined with modular software development, Cyclus 
is an easily transferable framework on which to build a strong fuel cycle simulation community.
%%%%%%%%%%%%%%%%%%%%%%%%%%%%%%%%%%
\subsection{Dynamic Module Capability}
Incorporation of dynamically loadable, independently construced modules is another key design concept in Cyclus. The core simulation engine 
comprising Cyclus is physically separate from the various modules determining the specific behavior that occurs in each simulation. 
Interaction between the core and modules occurs via dynamic linking, allowing for encapsulated development and providing developers the ability 
to focus on specific modules without involving the simulation engine. This increases efficiency and decreases the overall programming experience 
required. In the present study, indepenent modules were developed, such as the batch reactor and intelligent building region modules discussed 
above. This separation not only allows for multiple developers to work towards a common goal, e.g. fully describing a target scenario, but
also assists in separating simulation concerns into reasonably sized, independently testable modules.
%%%%%%%%%%%%%%%%%%%%%%%%%%%%%%%%%%
\subsection{Optimization}
The Cyclus project has recently added a linear, integer, and mixed integer program wrapper named Cyclopts \cite{cyclopts2012} to its tool set. Cyclopts relies on the
open-source solver Coin-or Branch and Cut \cite{coinCBC}, providing a simple interface to describe such optimization problems. At the present time,
developers have the ability to define a set of variables and to describe an objective function and series of constraints that utilize some subset
of those variables. Variables currently come in two flavors: integer and linear, corresponding to their respective solution techniques. Additional
work will be performed to add cutting plane support as well as a suite of unit tests.
%%%%%%%%%%%%%%%%%%%%%%%%%%%%%%%%%%%%%%%%%%%%%%%%%%%%%%%%%%%%%%%%%%%%%%%%%%%%%%%%
\section{Once-Through Fuel Cycle Benchmarks}
blah
\subsection{Scenarios}
blah
\subsection{Results \& Analysis}
blah

\bibliography{bibliography}
\end{document}

% What Cyclus is, who we are

% Cyclus Design Philosophy
% -RIF
% -Markets
% -Optimization

% Once Through Set-Up
% -modules used
% -describe integer program(s)
% -general decision making

% Scenarios Analyzed

% Conclusion

% ---------
% mixins - builder, producer, consumer
%  -- defines interface for that type of scenario interaction
