% Thanks to Seth Johnson (sethrj@umich.edu) for publishing his ans template

\documentclass{anstrans}
\bibliographystyle{ans}

%%%%%%%%%%%%%%%%%%%%%%%%%%%%%%%%%%%
\title{Once-Through Bechmarks with \emph{Cyclus}, a Modular, Open-Source Fuel Cycle Simulator}
\author{Matthew J.~Gidden, Paul P.H.~Wilson, Kathryn D.~Huff, Robert W.~Carlsen}

%% uncomment these next five only if using anstrans
\institute{Department of Nuclear Engineering \& Engineering Physics, University of Wisconsin - Madison, Madison, WI, 53703}
\email{gidden@wisc.edu}

\usepackage{times}
\usepackage{graphicx}
\usepackage{microtype} % if using PDF
\newcommand{\units}[1] {\:\text{#1}}%
\newcommand{\SN}{S$_N$}%{S$_\text{N}$}%{$S_N$}%

\date{2012/06/29}
%%%%%%%%%%%%%%%%%%%%%%%%%%%%%%%%%%%
\begin{document}

%%%%%%%%%%%%%%%%%%%%%%%%%%%%%%%%%%%%%%%%%%%%%%%%%%%%%%%%%%%%%%%%%%%%%%%%%%%%%%%%
\section{Introduction}
The \emph{Cyclus} project at the University of Wisconsin - Madison is the result of previous lessons learned and is designed to be a 
modular application, allowing for a variety of collaborators and a potentially diverse user-base. \emph{Cyclus} is written in C++ and
is composed of a core library which is linked to a selection of module and utility libraries. In its current form, \emph{Cyclus} uses
XML-based pre-processing and a SQL-based output database. The \emph{Cyclus} team has recently grown and now incorporates a variety of 
expertise: output visualization capability through collaboration with the University of Utah, server-client communication via the 
University of Idaho, input visualization and control with the University of Texas - Austin, and social communication expertise 
through collaborators at UW-Madison to assist the mission-critical goal of relevancy vis-\`{a}-vis policy makers. Accordingly, the 
\emph{Cyclus} project is expanding efforts in the realms of both structural capability and benchmarking calculations.

A series of once-through fuel cycle scenarios have been conducted using the \emph{Cyclus} core and accompanying modules. Where needed,
additional modules have been added to the basick pack of \emph{Cyclus} module, including a reactor module that operates in a batch 
mode and an region model that intelligently makes building decisions given a demand function. The results of these scenarios
are then compared with current industry standard simulators, such as VISION \cite{vision2009}, to provide a benchmark of the
\emph{Cyclus} results.
%%%%%%%%%%%%%%%%%%%%%%%%%%%%%%%%%%%%%%%%%%%%%%%%%%%%%%%%%%%%%%%%%%%%%%%%%%%%%%%%
\section{\emph{Cyclus} Design and Development}
The paradigm under which \emph{Cyclus} has been developed and is distributed is unique and offers a number of advantages regarding its 
potential wide-spread adoption by a diverse user base. \emph{Cyclus} is an open source collection of libraries that
allow for both out-of-the-box capability and a platform on which to develop additional modules. At present, the \emph{Cyclus} suite 
includes a core library, a basic module pack and utilities library, and an optimization library. Each can be linked to an 
application to be used in concert.
%%%%%%%%%%%%%%%%%%%%%%%%%%%%%%%%%%
\subsection{Open Development}
As one of the leading principles guiding the development of \emph{Cyclus}, an open-source repository provides a high degree of flexibility 
and a large amount of exposure to potential collaborators and developers. The \emph{Cyclus} repository is publicly available via 
GitHub \cite{cyclus2012}. A number of tools are available to \emph{Cyclus} developers in order to maintain software development best-practices, 
including distributed version control, automatic documentation, and in-depth issue management. Additionally, \emph{Cyclus} developers have taken 
advantage of existing, external open-source libraries in order to conform to current standards, including an expansion of the C++ standard
library and fully benchmarked linear and integer programming solvers. Perhaps the most important attribute of the open development paradigm 
is that it allows for unfettered access to the \emph{Cyclus} core and basic module source code. Combined with modular software development, \emph{Cyclus} 
is an easily transferable framework on which to build a strong fuel cycle simulation community.
%%%%%%%%%%%%%%%%%%%%%%%%%%%%%%%%%%
\subsection{Dynamic Module Capability}
Incorporation of dynamically loadable, independently construced modules is another key design concept in \emph{Cyclus}. The core simulation engine 
comprising \emph{Cyclus} is physically separate from the various modules determining the specific behavior that occurs in each simulation. 
Interaction between the core and modules occurs via dynamic linking, allowing for encapsulated development and providing developers the ability 
to focus on specific modules without involving the simulation engine. This increases efficiency and decreases the overall programming experience 
required. In the present study, indepenent modules were developed, such as the batch reactor and intelligent building region modules discussed 
above. This separation not only allows for multiple developers to work towards a common goal, e.g. fully describing a target scenario, but
also assists in separating simulation concerns into reasonably sized, independently testable modules.
%%%%%%%%%%%%%%%%%%%%%%%%%%%%%%%%%%
\subsection{Optimization}
The \emph{Cyclus} project has recently added a linear, integer, and mixed integer program wrapper named Cyclopts \cite{cyclopts2012} to its tool set. Cyclopts relies on the
open-source solver Coin-or Branch and Cut \cite{coinCBC}, providing a simple interface to describe such optimization problems. At the present time,
developers have the ability to define a set of variables and to describe an objective function and series of constraints that utilize some subset
of those variables. Variables currently come in two flavors: integer and linear, corresponding to their respective solution techniques. Additional
work will be performed to add cutting plane support as well as a suite of unit tests.
%%%%%%%%%%%%%%%%%%%%%%%%%%%%%%%%%%%%%%%%%%%%%%%%%%%%%%%%%%%%%%%%%%%%%%%%%%%%%%%%
\section{Once-Through Fuel Cycle Benchmarks}
In order to model a once-through fuel cycle, a subset of real-world facility interactions were chosen to be analyzed. Among the chosen facilities are:
mines, enrichment facilities, reactors, and storage facilities. In order to drive the simulation, an electricity demand function is provided to the 
simulation which determines the supply requirement for reactors at a given time step. Once the simulation is complete, an analysis of material
flows and facility deployment as a function of time is provided. Of specific concern with respect to the once-through fuel cycle is long term used
supply and the corresponding storage requirements.
%%%%%%%%%%%%%%%%%%%%%%%%%%%%%%%%%%
\subsection{Scenario Construction}
A number of modules have been developed to perform this analysis. Reactor deployment decisions are driven by the GrowthRegion class. An electricity demand is
defined for that region, either as a linear or exponential function. The region is also provided with a set of facilities (reactors) that can meet
this demand. At any timestep in which there exists a demand gap, i.e. there exists more demand than supply, a build decision is made. This decision is modeled
as the following integer program:
\begin{subequations} \label{eqs:optBuild}
\begin{equation} \label{eq:optBuildObj}
  \min \sum_{i=1}^{N}n_i*c_i
\end{equation}
\begin{equation} \label{eq:optBuildConst}
  s.t. \, \sum_{i=1}^{N}n_i*\phi_i \ge \Phi
\end{equation}
\begin{equation} \label{eq:optBuildBounds}
  n_i \in [0,\infty) \, \forall \, i \in I, \: n_i \: integer
\end{equation}
where $\Phi$ is the demand to be met, $I$ is the set of facilities meeting the demand, and, for each facility, $c_i$ is the cost, and $\phi_i$ is the capacity. 
Finally, $n_i$ is the optimized number of facilities to build.
\end{subequations}

While electricity demand directly corresponds to reactor deployment, supporting facilities, e.g. enrichment facilities, mines, and storage facilities, depend
on overal all reactor fuel input and output requirements. There are multiple strategies that can be used to determine supporting facility deployment, of which
a minimal constraint approach is first analyzed followed by a supply-demand constraint approach, i.e. when the demand for a resource (say, fresh fuel) reaches
a percentage of the total supply, a facility meeting that demand is deployed.

\subsection{Benchmarking and Parameter Variation}
In order to benchmark simulation results, common once-through scenarios have been defined that can be modeled by both \emph{Cyclus} and VISION. VISION uses an Excel 
Spreadsheet input paradigm which allows one to vary a number of parameters. ``Base Case Scenarios'' are also defined, among which is a once-through scenario.
Accordingly, this provided scenario is used and some of its corresponding parameters are varied in order to define a relative benchmark between \emph{Cyclus} and
VISION. Specifically, the following parameters are varied:
\begin{itemize}
\item number of reactor types (e.g. LWRs and HWRs)
\item electricity demand rate
\item storage availability date (i.e. when a repository is available)
\end{itemize}
%%%%%%%%%%%%%%%%%%%%%%%%%%%%%%%%%%
\section{Results \& Analysis}
blah
%%%%%%%%%%%%%%%%%%%%%%%%%%%%%%%%%%%%%%%%%%%%%%%%%%%%%%%%%%%%%%%%%%%%%%%%%%%%%%%%
\section{Acknowlegements}
This research is being performed using funding received from the DOE Office of Nuclear Energy's Nuclear Energy University Programs. 
The author thanks the NEUP for its generous support.
%%%%%%%%%%%%%%%%%%%%%%%%%%%%%%%%%%%%%%%%%%%%%%%%%%%%%%%%%%%%%%%%%%%%%%%%%%%%%%%%
\bibliography{bibliography}
\end{document}
