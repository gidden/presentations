% Overview : intro.tex
% Explain why this talk is being given.

\begin{frame}[ctb!]
  \frametitle{Introduction : Purpose}
  Fuel cycle simulators are designed to answer policy-related questions
  regarding transitions from one equilibrium state to another.

  \vspace{0.2cm}

  \pause
  A simulator answers the following questions as a function of its 
  parameter space:
  \begin{itemize}
    \item how much material exists
    \item where does that material reside
    \item from/to where and when is material transported
    \item what kinds of facilities are needed
    \item when is each type of facility needed
  \end{itemize}
\end{frame}

\begin{frame}[ctb!]
  \frametitle{Introduction : VISION}
  VISION is one such simulator developed at INL and used by the DOE. 
  It is well represented in the literature and can model most aspects 
  of the fuel cycle. \cite{yacout_vision_2006}
  \begin{itemize}
    \item continuous material flows
    \item fleet-based facility deployment
    \item some regional modeling capability
    \item input/output via Excel
    \item simulation engine via Powersim
  \end{itemize}
\end{frame}

\begin{frame}[ctb!]
  \frametitle{Introduction : Cyclus}
  There have been many implementations of fuel cycle simulators. 
  Cyclus is designed to provide a common fuel cycle simulator 
  framework, i.e. a common simulator language.

  \vspace{0.2cm}

  In this vein, we wish to show similar capabilities to other 
  simulators, e.g. VISION.

  \vspace{0.2cm}

  \pause
  This discussion will showcase capability additions to the Cyclus
  simulation engine and the modules that use them. We then provide a
  benchmark example of VISION and Cyclus output.
\end{frame}